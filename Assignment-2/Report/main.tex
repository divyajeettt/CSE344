\documentclass[10pt]{article}
\usepackage[utf8]{inputenc}
\usepackage{geometry}
\usepackage{amsfonts}
\usepackage{hyperref}
\usepackage{enumitem}
\usepackage{graphicx}
\usepackage{tabularx}
\usepackage{amsmath}
\usepackage{xcolor}


\newcommand{\unit}[1]{\mathbf{\hat{#1}}}


\title{
    \textbf{CSE344: Computer Vision} \\ \vspace*{-5pt}
    \textbf{\large{Assignment-2}}
}

\author{\href{mailto:divyajeet21529@iiitd.ac.in}{Divyajeet Singh (2021529)}}
\date{\today}

\geometry{a4paper, left=20mm, right=20mm, top=20mm, bottom=20mm}


\begin{document}
    \maketitle

    \section*{\textbf{Question 1.}}
    \begin{enumerate}
        \item The given transformation is a composition of the following three transformations
        (in order):
        \begin{enumerate}[label=(\alph*)]
            \item Rotation by $\frac{\pi}{2}$ about the $Y$-axis
            \item Rotation by $\frac{-\pi}{2}$ about the $X$-axis
            \item Translation by $\begin{bmatrix} -1 & 3 & 2 \end{bmatrix}^{\top}$
        \end{enumerate}
        The coordinate transformation matrices (using 3-dimensional homogeneous coordinates),
        for the three transformations are
        \begin{align*}
            R_{y}\left(\frac{\pi}{2}\right) &= \begin{bmatrix}
                \cos{\left(\frac{\pi}{2}\right)} & 0 & \sin{\left(\frac{\pi}{2}\right)} & 0 \\
                0 & 1 & 0 & 0 \\
                -\sin{\left(\frac{\pi}{2}\right)} & 0 & \cos{\left(\frac{\pi}{2}\right)} & 0 \\
                0 & 0 & 0 & 1
            \end{bmatrix} = \begin{bmatrix}
                0 & 0 & 1 & 0 \\
                0 & 1 & 0 & 0 \\
                -1 & 0 & 0 & 0 \\
                0 & 0 & 0 & 1
            \end{bmatrix} \\
            R_{x}\left(\frac{-\pi}{2}\right) &= \begin{bmatrix}
                1 & 0 & 0 & 0 \\
                0 & \cos{\left(\frac{-\pi}{2}\right)} & -\sin{\left(\frac{-\pi}{2}\right)} & 0 \\
                0 & \sin{\left(\frac{-\pi}{2}\right)} & \cos{\left(\frac{-\pi}{2}\right)} & 0 \\
                0 & 0 & 0 & 1
            \end{bmatrix} = \begin{bmatrix}
                1 & 0 & 0 & 0 \\
                0 & 0 & 1 & 0 \\
                0 & -1 & 0 & 0 \\
                0 & 0 & 0 & 1
            \end{bmatrix} \\
            T\left(\begin{bmatrix} -1 \\ 3 \\ 2 \end{bmatrix}\right) &= \begin{bmatrix}
                1 & 0 & 0 & -1 \\
                0 & 1 & 0 & 3 \\
                0 & 0 & 1 & 2 \\
                0 & 0 & 0 & 1
            \end{bmatrix}
        \end{align*}
        The coordinate transformation matrix for the full transformation is then given by
        \begin{align*}
            T &= T\left(\begin{bmatrix} -1 \\ 3 \\ 2 \end{bmatrix}\right) \circ
                R_{x}\left(\frac{-\pi}{2}\right) \circ R_{y}\left(\frac{\pi}{2}\right) \\
            &= T\left(\begin{bmatrix} -1 \\ 3 \\ 2 \end{bmatrix}\right) \begin{bmatrix}
                1 & 0 & 0 & 0 \\
                0 & 0 & 1 & 0 \\
                0 & -1 & 0 & 0 \\
                0 & 0 & 0 & 1
            \end{bmatrix} \begin{bmatrix}
                0 & 0 & 1 & 0 \\
                0 & 1 & 0 & 0 \\
                -1 & 0 & 0 & 0 \\
                0 & 0 & 0 & 1
            \end{bmatrix} \\
            &= \begin{bmatrix}
                1 & 0 & 0 & -1 \\
                0 & 1 & 0 & 3 \\
                0 & 0 & 1 & 2 \\
                0 & 0 & 0 & 1
            \end{bmatrix} \begin{bmatrix}
                0 & 0 & 1 & 0 \\
                -1 & 0 & 0 & 0 \\
                0 & -1 & 0 & 0 \\
                0 & 0 & 0 & 1
            \end{bmatrix} = \begin{bmatrix}
                0 & 0 & 1 & -1 \\
                -1 & 0 & 0 & 3 \\
                0 & -1 & 0 & 2 \\
                0 & 0 & 0 & 1
            \end{bmatrix}
        \end{align*}
        since rotations and translations are linear transformations in homogeneous
        coordinate systems, and hence composition of transformations is equivalent
        to multiplication of the transformation matrices.

        \item Now, we find the new coordinates of a given vector,
        $v = \begin{bmatrix} 2 & 5 & 1 \end{bmatrix}^{\top}$, after the transformation.
        The new coordinates, say $v'$, are given by simply applying the transformation
        matrix to $v$ in homogeneous coordinates, which gives
        \begin{equation*}
            v' = T v = \begin{bmatrix}
                0 & 0 & 1 & -1 \\
                -1 & 0 & 0 & 3 \\
                0 & -1 & 0 & 2 \\
                0 & 0 & 0 & 1
            \end{bmatrix} \begin{bmatrix} 2 \\ 5 \\ 1 \\ 1 \end{bmatrix} = \begin{bmatrix}
                0 \\
                1 \\
                -3 \\
                1
            \end{bmatrix}
        \end{equation*}
        So, the new coordinates of the given vector are
        $v' = \begin{bmatrix} 0 & 1 & -3 \end{bmatrix}^{\top}$. We also find the point
        that the origin of the initial frame of reference gets mapped to. For this, we
        simply apply the transformation matrix $T$ to the origin $\mathbf{0}$ in homogeneous
        coordinates, which gives
        \begin{equation*}
            T \mathbf{0} = \begin{bmatrix}
                0 & 0 & 1 & -1 \\
                -1 & 0 & 0 & 3 \\
                0 & -1 & 0 & 2 \\
                0 & 0 & 0 & 1
            \end{bmatrix} \begin{bmatrix} 0 \\ 0 \\ 0 \\ 1 \end{bmatrix} = \begin{bmatrix}
                -1 \\
                3 \\
                2 \\
                1
            \end{bmatrix}
        \end{equation*}
        So, the origin gets mapped to the point $\begin{bmatrix} -1 & 3 & 2 \end{bmatrix}^{\top}$.

        \item We now have the combined rotation matrix, $\mathbf{R}$ (written without homogeneous
        coordinates), as
        \begin{equation*}
            \mathbf{R} = \begin{bmatrix}
                0 & 0 & 1 \\
                -1 & 0 & 0 \\
                0 & -1 & 0
            \end{bmatrix}
        \end{equation*}
        Using Rodrigues formula, we can obtain the direction of the axis of this combined
        rotation in the original frame of reference and the angle of rotation about this axis.
        According to the Rodrigues formula, the angle of rotation is given by
        \begin{equation*}
            \theta = \cos^{-1}{\left(\frac{\textsc{Trace}(\mathbf{R}) - 1}{2}\right)}
            = \cos^{-1}{\left(\frac{-1}{2}\right)} = \frac{2\pi}{3}
        \end{equation*}
        The axis of rotation, $\unit{n}$, is given by
        \begin{equation*}
            \unit{n} = \frac{1}{2\sin{\theta}} \begin{bmatrix}
                \mathbf{R}_{32} - \mathbf{R}_{23} \\
                \mathbf{R}_{13} - \mathbf{R}_{31} \\
                \mathbf{R}_{21} - \mathbf{R}_{12}
            \end{bmatrix}
            = \frac{1}{2\sin{\frac{2\pi}{3}}} \begin{bmatrix}
                -1 - 0 \\
                1 - 0 \\
                -1 - 0
            \end{bmatrix} = \frac{1}{\sqrt{3}} \begin{bmatrix}
                -1 \\
                1 \\
                -1
            \end{bmatrix}
        \end{equation*}
        Therefore, the combined rotation is a rotation of $\frac{2\pi}{3}$ about the axis
        $\frac{1}{\sqrt{3}} \begin{bmatrix} -1 & 1 & -1 \end{bmatrix}^{\top}$,
        i.e.
        \begin{equation*}
            \mathbf{R} \equiv R\left(\frac{1}{\sqrt{3}} \begin{bmatrix}
                -1  \\
                1 \\
                -1
            \end{bmatrix}, \frac{2\pi}{3}\right)
        \end{equation*}

        \item We now use the above axis $\unit{n}$ and angle $\theta$ to calculate the
        the rotation matrix (say) $\mathbf{R'}$ for the rotation, and show that it is the
        same as matrix $\mathbf{R}$ that we obtained through sequentially applying the two
        given rotations. Using the Rodrigues formula, the rotation matrix $\mathbf{R'}$ is
        given by
        \begin{equation*}
            \mathbf{R'} = \mathbf{I} + \sin{\theta} \mathbf{N} + (1 - \cos{\theta}) \mathbf{N}^2
            \quad \text{where} \quad \mathbf{N} = \begin{bmatrix}
                0 & -n_{3} & n_{2} \\
                n_{3} & 0 & -n_{1} \\
                -n_{2} & n_{1} & 0
            \end{bmatrix} \quad (n_{i} \text{ represent the components of } \unit{n})
        \end{equation*}
        We first find $\mathbf{N}$ and then use it to find $\mathbf{R'}$. We have
        \begin{equation*}
            \mathbf{N} =  \frac{1}{\sqrt{3}} \begin{bmatrix}
                0 & 1 & 1 \\
                -1 & 0 & 1 \\
                -1 & -1 & 0
            \end{bmatrix} \implies \mathbf{N}^{2} = \frac{1}{3} \begin{bmatrix}
                -2 & -1 & 1 \\
                -1 & -2 & -1 \\
                1 & -1 & -2
            \end{bmatrix} \\
        \end{equation*}
        Using these, we find
        \begin{align*}
            \mathbf{R'} &= \begin{bmatrix}
                1 & 0 & 0 \\
                0 & 1 & 0 \\
                0 & 0 & 1
            \end{bmatrix} + \frac{1}{\sqrt{3}} \sin{\frac{2\pi}{3}} \begin{bmatrix}
                0 & 1 & 1 \\
                -1 & 0 & 1 \\
                -1 & -1 & 0
            \end{bmatrix} + \frac{1}{3} \left(1 - \cos{\frac{2\pi}{3}} \right) \begin{bmatrix}
                -2 & -1 & 1 \\
                -1 & -2 & -1 \\
                1 & -1 & -2
            \end{bmatrix} \\
            &= \begin{bmatrix}
                1 & 0 & 0 \\
                0 & 1 & 0 \\
                0 & 0 & 1
            \end{bmatrix} + \frac{1}{2} \begin{bmatrix}
                0 & 1 & 1 \\
                -1 & 0 & 1 \\
                -1 & -1 & 0
            \end{bmatrix} + \frac{1}{6} \begin{bmatrix}
                -2 & -1 & 1 \\
                -1 & -2 & -1 \\
                1 & -1 & -2
            \end{bmatrix} \\
            &= \begin{bmatrix}
                0 & 0 & 1 \\
                -1 & 0 & 0 \\
                0 & -1 & 0
            \end{bmatrix} = \mathbf{R}
        \end{align*}
        Hence, we have shown that the rotation matrix $\mathbf{R'}$ obtained using the axis
        $\unit{n}$ and angle $\theta$ is the same as the matrix $\mathbf{R}$ that we obtained
        through sequentially applying the two given rotations. This also proves the correctness
        of the axis and angle obtained using the Rodrigues formula.
    \end{enumerate}

    \section*{\textbf{Question 2.}}
    By Rodrigues formula, we know that the rotated vector for the given rotation is
    given by
    \begin{align*}
        \mathbf{R} \mathbf{x} &= \mathbf{x} + (\unit{u} \times \mathbf{x}) \sin{\theta}
        + (1 - \cos{\theta}) \unit{u} \times (\unit{u} \times \mathbf{x}) \\
        &= \mathbf{x} + (\unit{u} \times \mathbf{x}) \sin{\theta}
        + (1 - \cos{\theta}) \left[ (\unit{u}^{\top} \mathbf{x}) \unit{u} - (\unit{u}^{\top}
        \unit{u}) \mathbf{x} \right] \\
        &= \mathbf{x} + (\unit{u} \times \mathbf{x}) \sin{\theta}
        + (1 - \cos{\theta}) (\unit{u}^{\top} \mathbf{x}) \unit{u} - (1 - \cos{\theta})
        \mathbf{x} \\
        &= \mathbf{x} - \mathbf{x} + \mathbf{x} \cos{\theta} + (\unit{u} \times \mathbf{x})
        \sin{\theta} + (1 - \cos{\theta}) (\unit{u}^{\top} \mathbf{x}) \unit{u} \\
        &= \mathbf{x} \cos{\theta} + (\unit{u} \times \mathbf{x}) \sin{\theta}
        + (\unit{u}^{\top} \mathbf{x}) (1 - \cos{\theta}) \unit{u} \tag*{$\square$}
    \end{align*}
    which proves the result, using the vector triple product identity that states
    for any vectors $\mathbf{a}$, $\mathbf{b}$, and $\mathbf{c}$,
    \begin{equation*}
        \mathbf{a} \times (\mathbf{b} \times \mathbf{c}) = (\mathbf{a}^{\top} \mathbf{c})
        \mathbf{b} - (\mathbf{a}^{\top} \mathbf{b}) \mathbf{c}
    \end{equation*}
    and that $\unit{u}^{\top} \unit{u} = 1$ since $\unit{u}$ is a unit vector.

    \section*{\textbf{Question 3.}}
    We are given image formation equations of two cameras, $\mathbf{C}_{1}$ and
    $\mathbf{C}_{2}$, for a 3D point $\mathbf{X}$ in the world, as follows
    \begin{align*}
        \mathbf{x}_{1} &= \mathbf{K}_{1} \left[ \mathbf{R}_{1} \mid \mathbf{t}_{1} \right] \mathbf{X} \\
        \mathbf{x}_{2} &= \mathbf{K}_{2} \left[ \mathbf{R}_{2} \mid \mathbf{t}_{2} \right] \mathbf{X}
    \end{align*}
    We are given that the first camera frame of reference is known and is the world
    coordinate frame. $\mathbf{C}_{2}$ is obtained by a pure 3D rotation $\mathbf{R}$
    applied to $\mathbf{C}_{1}$. We need to show that the image points $\mathbf{x}_{1}$
    and $\mathbf{x}_{2}$ are related by
    \begin{equation*}
        \mathbf{x}_{1} = \mathbf{H} \mathbf{x}_{2}
    \end{equation*}
    for some invertible $\mathbf{H}_{3 \times 3}$ and find it in terms of $\mathbf{K}_{1}$,
    $\mathbf{K}_{2}$, and $\mathbf{R}$. \\
    We know that $\mathbf{C}_{2}$ is obtained by a 3D rotation $\mathbf{R}$ applied to
    $\mathbf{C}_{1}$. This means that $\mathbf{C}_{1}$ can be obtained by applying the
    inverse rotation of $\mathbf{R}$ to $\mathbf{C}_{2}$. Since $\mathbf{R}$ is an orthogonal
    matrix, $\mathbf{R}^{-1} = \mathbf{R}^{\top}$.

\end{document}